\documentclass[a4paper]{book}
\usepackage[times,inconsolata,hyper]{Rd}
\usepackage{makeidx}
\usepackage[utf8]{inputenc} % @SET ENCODING@
% \usepackage{graphicx} % @USE GRAPHICX@
\makeindex{}
\begin{document}
\chapter*{}
\begin{center}
{\textbf{\huge Package `bbw'}}
\par\bigskip{\large \today}
\end{center}
\begin{description}
\raggedright{}
\inputencoding{utf8}
\item[Type]\AsIs{Package}
\item[Title]\AsIs{Blocked Weighted Bootstrap}
\item[Version]\AsIs{0.1.2}
\item[Description]\AsIs{The blocked weighted bootstrap (BBW) is an estimation technique
for use with data from two-stage cluster sampled surveys in which either
prior weighting (e.g. population-proportional sampling or PPS as used in
Standardized Monitoring and Assessment of Relief and Transitions or SMART
surveys) or posterior weighting (e.g. as used in rapid assessment method or
RAM and simple spatial sampling method or S3M surveys). The method was
developed by Accion Contra la Faim, Brixton Health, Concern Worldwide,
Global Alliance for Improved Nutrition, UNICEF Sierra Leone,  UNICEF Sudan
and Valid International. It has been tested by the Centers for Disease
Control (CDC) using infant and young child feeding (IYCF) data. See Cameron
et al (2008) <doi.org/10.1162/rest.90.3.414> for application of bootstrap
to cluster samples. See Aaron et al (2016) <doi.org/10.1371/journal.pone.0163176>
and Aaron et al (2016) <doi.org/10.1371/journal.pone.0162462> for application
of the blocked weighted bootstrap to estimate indicators from two-stage
cluster sampled surveys.}
\item[Imports]\AsIs{car}
\item[Depends]\AsIs{R (>= 3.0.1)}
\item[Suggests]\AsIs{knitr,
rmarkdown,
testthat}
\item[License]\AsIs{AGPL-3}
\item[Encoding]\AsIs{UTF-8}
\item[LazyData]\AsIs{true}
\item[RoxygenNote]\AsIs{6.0.1}
\item[URL]\AsIs{}\url{https://github.com/validmeasures/bbw}\AsIs{}
\item[BugReports]\AsIs{}\url{https://github.com/validmeasures/bbw/issues}\AsIs{}
\item[VignetteBuilder]\AsIs{knitr}
\end{description}
\Rdcontents{\R{} topics documented:}
\inputencoding{utf8}
\HeaderA{bbw}{Blocked Weighted Bootstrap}{bbw}
\aliasA{bbw-package}{bbw}{bbw.Rdash.package}
%
\begin{Description}\relax
The \code{blocked weighted bootstrap (BBW)} is an estimation technique for
use with data from two-stage cluster sampled surveys in which either prior
weighting (e.g. \code{population proportional sampling} or \code{PPS} as
used in \code{Standardized Monitoring and Assessment of Relief and
Transitions} or \code{SMART} surveys) or posterior weighting (e.g. as used in
\code{Rapid Assessment Method} or \code{RAM} and \code{Simple Spatial
Sampling Method} or \code{S3M} surveys). The method was developed by
Accion contra la Faim (ACF), Brixton Health, Concern Worldwide, Global
Alliance for Improved Nutrition (GAIN), UNICEF Sierra Leone, UNICEF Sudan and
Valid International. It has been tested by the Centers for Disease
Control (CDC) using infant and young child feeding (IYCF) data.
\end{Description}
%
\begin{Details}\relax
The bootstrap technique is summarised in this
\Rhref{https://en.wikipedia.org/wiki/Bootstrapping_(statistics)}{article}.
The \code{BBW} used in \code{RAM} and \code{S3M} is a modification to the
\code{percentile bootstrap} to include \code{blocking} and \code{weighting} to
account for a complex sample design.

With \code{RAM} and \code{S3M} surveys, the sample is complex in the sense
that it is an unweighted cluster sample. Data analysis procedures need to
account for the sample design. A \code{blocked weighted bootstrap (BBW)}
can be used:
\begin{description}

\item[\code{Blocked}] The block corresponds to the primary sampling unit
(\eqn{PSU = cluster}{}). \code{PSUs} are resampled with replacement.
Observations within the resampled \code{PSUs} are also sampled with
replacement.

\item[\code{Weighted}] \code{RAM} and \code{S3M} samples do not use
\code{population proportional sampling (PPS)} to weight the sample prior to
data collection (e.g. as is done with \code{SMART} surveys). This means that
a posterior weighting procedure is required. \code{BBW} uses a
\code{"roulette wheel"} algorithm to weight (i.e. by population) the
selection probability of \code{PSUs} in bootstrap replicates.

\end{description}


In the case of prior weighting by \code{PPS} all clusters are given the
same weight. With posterior weighting (as in \code{RAM} or \code{S3M})
the weight is the population of each \code{PSU}. This procedure is very
similar to the \Rhref{https://en.wikipedia.org/wiki/Fitness_proportionate_selection}{fitness proportionate selection}
technique used in evolutionary computing.

A total of \eqn{m}{} \code{PSUs} are sampled with replacement for each
bootstrap replicate (where \eqn{m}{} is the number of \code{PSUs} in the survey
sample).

The required statistic is applied to each replicate. The reported estimate
consists of the \code{0.025th (95\% LCL)}, \code{0.5th (point estimate)}, and
\code{0.975th (95\% UCL)} quantiles of the distribution of the statistic across
all survey replicates.

Early versions of the \code{BBW} did not resample observations within
\code{PSUs} following:

\Cite{Cameron AC, Gelbach JB, Miller DL, Bootstrap-based improvements for
inference with clustered errors, Review of Economics and Statistics,
2008:90;414–427 \url{doi.org/10.1162/rest.90.3.414}}

and used a large number (e.g. \eqn{3999}{}) survey replicates. Current versions of
the \code{BBW} resample observations within \code{PSUs} and use a smaller
number of survey replicates (e.g. \eqn{n = 400}{}). This is a more computationally
efficient approach
\end{Details}
\inputencoding{utf8}
\HeaderA{bootBW}{\code{bootBW} : Blocked Weighted Bootstrap}{bootBW}
%
\begin{Description}\relax
The \code{blocked weighted bootstrap (BBW)} is an estimation technique for
use with data from two-stage cluster sampled surveys in which either prior
weighting (e.g. \code{population proportional sampling} or \code{PPS} as
used in \code{SMART} surveys) or posterior weighting (e.g. as used in
\code{RAM} and \code{S3M} surveys).
\end{Description}
%
\begin{Usage}
\begin{verbatim}
bootBW(x, w, statistic, params, outputColumns, replicates = 400)
\end{verbatim}
\end{Usage}
%
\begin{Arguments}
\begin{ldescription}
\item[\code{x}] A data frame with primary sampling unit (PSU) in column named \code{psu}

\item[\code{w}] A data frame with primary sampling unit (PSU) in column named \code{psu}
and survey weight (i.e. PSU population) in column named \code{pop}

\item[\code{statistic}] A function operating on data in \code{x} (see Example)

\item[\code{params}] Parameters (named columns in \code{x}) passed to the function
specified in \code{statistic}

\item[\code{outputColumns}] Names of columns in output data frame

\item[\code{replicates}] Number of bootstrap replicates
\end{ldescription}
\end{Arguments}
%
\begin{Value}
A data frame with:
\begin{description}

\item[] \code{ncol} = length(outputColumns)
\item[] \code{nrow} = replicates
\item[] \code{names} = outputColumns

\end{description}

\end{Value}
%
\begin{Examples}
\begin{ExampleCode}
# Example function - estimate a proportion for a binary (0/1) variable):

oneP <- function(x, params) {
  v1 <- params[1]
  v1Data <- x[[v1]]
  oneP <- mean(v1Data, na.rm = TRUE)
  return(oneP)
}

# Example call to bootBW function using RAM-OP test data:

bootP <- bootBW(x = indicatorsHH,
                w = villageData,
                statistic = oneP,
                params = "anc1",
                outputColumns = "anc1",
                replicates = 9)

# Example estimate with 95% CI:

quantile(bootP, probs = c(0.500, 0.025, 0.975), na.rm = TRUE)

\end{ExampleCode}
\end{Examples}
\inputencoding{utf8}
\HeaderA{bootClassic}{bootClassic}{bootClassic}
%
\begin{Description}\relax
Simple proportion statistics function for bootstrap estimation
\end{Description}
%
\begin{Usage}
\begin{verbatim}
bootClassic(x, params)
\end{verbatim}
\end{Usage}
%
\begin{Arguments}
\begin{ldescription}
\item[\code{x}] A data frame with \code{primary sampling unit (PSU)} in column named
\code{psu} and with data column/s containing the binary variable/s (0/1) of
interest with column names corresponding to \code{params} values

\item[\code{params}] A vector of column names corresponding to the binary variables
of interest contained in \code{x}
\end{ldescription}
\end{Arguments}
%
\begin{Value}
A numeric vector of the mean of each binary variable of interest with
length equal to \code{length(params)}
\end{Value}
%
\begin{Examples}
\begin{ExampleCode}

# Example call to bootClassic function

meanResults <- bootClassic(x = indicatorsHH,
                           params = "anc1")

\end{ExampleCode}
\end{Examples}
\inputencoding{utf8}
\HeaderA{bootPROBIT}{bootPROBIT}{bootPROBIT}
%
\begin{Description}\relax
PROBIT statistics function for bootstrap estimation
\end{Description}
%
\begin{Usage}
\begin{verbatim}
bootPROBIT(x, params, threshold = THRESHOLD)
\end{verbatim}
\end{Usage}
%
\begin{Arguments}
\begin{ldescription}
\item[\code{x}] A data frame with \code{primary sampling unit (PSU)} in column named
\code{psu} and with data column/s containing the continuous variable/s of
interest with column names corresponding to \code{params} values

\item[\code{params}] A vector of column names corresponding to the continuous
variables of interest contained in \code{x}

\item[\code{threshold}] cut-off value for continuous variable to differentiate
case and non-case
\end{ldescription}
\end{Arguments}
%
\begin{Value}
A numeric vector of the PROBIT estimate of each continuous variable
of interest with length equal to \code{length(params)}
\end{Value}
%
\begin{Examples}
\begin{ExampleCode}

# Example call to bootBW function:

bootPROBIT(x = indicatorsCH1,
           params = "muac1",
           threshold = 115)

\end{ExampleCode}
\end{Examples}
\inputencoding{utf8}
\HeaderA{indicatorsCH1}{Child Morbidity, Health Service Coverage, Anthropometry}{indicatorsCH1}
\keyword{datasets}{indicatorsCH1}
%
\begin{Description}\relax
Child indicators on morbidity, health service coverage and anthropometry
calculated from survey data collected in survey conducted in 4 districts
from 3 regions in Somalia.
\end{Description}
%
\begin{Usage}
\begin{verbatim}
indicatorsCH1
\end{verbatim}
\end{Usage}
%
\begin{Format}
A data frame with 14 columns and 3090 rows.
\begin{description}

\item[\code{psu}] The PSU identifier. This must use the same coding system
used to identify the PSUs that is used in the indicators dataset
\item[\code{mID}] The mother identifier
\item[\code{cID}] The child identifier
\item[\code{ch1}] Diarrhoea in the past 2 weeks (0/1)
\item[\code{ch2}] Fever in the past 2 weeks (0/1)
\item[\code{ch3}] Cough in the past 2 weeks (0/1)
\item[\code{ch4}] Immunisation card (0/1)
\item[\code{ch5}] BCG immunisation (0/1)
\item[\code{ch6}] Vitamin A coverage in the past month (0/1)
\item[\code{ch7}] Anti-helminth coverage in the past month (0/1)
\item[\code{sex}] Sex of child
\item[\code{muac1}] Mid-upper arm circumference in mm
\item[\code{muac2}] Mid-upper arm circumference in mm
\item[\code{oedema}] Oedema (0/1)

\end{description}
\end{Format}
%
\begin{Source}\relax
Mother and child health and nutrition survey in 3 regions of Somalia
\end{Source}
\inputencoding{utf8}
\HeaderA{indicatorsCH2}{Infant and Child Feeding Index}{indicatorsCH2}
\keyword{datasets}{indicatorsCH2}
%
\begin{Description}\relax
Infant and young child feeding indicators using the infant and child feeding
index (ICFI) by Arimond and Ruel. Calculated from survey data collected in
survey conducted in 4 districts from 3 regions in Somalia.
\end{Description}
%
\begin{Usage}
\begin{verbatim}
indicatorsCH2
\end{verbatim}
\end{Usage}
%
\begin{Format}
A data frame with 13 columns and 2083 rows.
\begin{description}

\item[\code{psu}] The PSU identifier. This must use the same coding system
used to identify the PSUs that is used in the indicators dataset
\item[\code{mID}] The mother identifier
\item[\code{cID}] The child identifier
\item[\code{ebf}] Exclusive breastfeeding (0/1)
\item[\code{cbf}] Continued breastfeeding (0/1)
\item[\code{ddd}] Dietary diversity (0/1)
\item[\code{mfd}] Meal frequency (0/1)
\item[\code{icfi}] Infant and child feeding index (from 0 to 6)
\item[\code{iycf}] Good IYCF
\item[\code{icfiProp}] Good ICFI
\item[\code{age}] Child's age
\item[\code{bf}] Child is breastfeeding (0/1)
\item[\code{bfStop}] Age in months child stopped breastfeeding

\end{description}
\end{Format}
%
\begin{Source}\relax
Mother and child health and nutrition survey in 3 regions of Somalia
\end{Source}
\inputencoding{utf8}
\HeaderA{indicatorsHH}{Mother Indicators Dataset}{indicatorsHH}
\keyword{datasets}{indicatorsHH}
%
\begin{Description}\relax
Mother indicators for health and nutrition calculated from survey data
collected in survey conducted in 4 districts from 3 regions in Somalia.
\end{Description}
%
\begin{Usage}
\begin{verbatim}
indicatorsHH
\end{verbatim}
\end{Usage}
%
\begin{Format}
A data frame with 24 columns and 2136 rows:
\begin{description}

\item[\code{psu}] The PSU identifier. This must use the same coding system
used to identify the PSUs that is used in the indicators dataset
\item[\code{mID}] The mother identifier
\item[\code{mMUAC}] Mothers with mid-upper arm circumference < 230 mm (0/1)
\item[\code{anc1}] At least 1 antenatal care visit with a trained health
professional (0/1)
\item[\code{anc2}] At least 4 antenatal care visits with any service
provider (0/1)
\item[\code{anc3}] FeFol coverage (0/1)
\item[\code{anc4}] Vitamin A coverage (0/1)
\item[\code{wash1}] Improved sources of drinking water (0/1)
\item[\code{wash2}] Improved sources of other water (0/1)
\item[\code{wash3}] Probable safe drinking water (0/1)
\item[\code{wash4}] Number of litres of water collected in a day
\item[\code{wash5}] Improved toilet facilities (0/1)
\item[\code{wash6}] Human waste disposal practices / behaviour (0/1)
\item[\code{wash7a}] Handwashing score (from 0 to 5)
\item[\code{wash7b}] Handwashing score of 5 (0/1)
\item[\code{hhs1}] Household hunger score (from 0 to 6)
\item[\code{hhs2}] Little or no hunger (0/1)
\item[\code{hhs3}] Moderate hunger (0/1)
\item[\code{hhs4}] Severe hunger (0/1)
\item[\code{mfg}] Mother's dietary diversity score
\item[\code{pVitA}] Plant-based vitamin A-rich foods (0/1)
\item[\code{aVitA}] Animal-based vitamin A-rich foods (0/1)
\item[\code{xVitA}] Any vitamin A-rich foods (0/1)
\item[\code{iron}] Iron-rich foods (0/1)

\end{description}
\end{Format}
%
\begin{Source}\relax
Mother and child health and nutrition survey in 3 regions of Somalia
\end{Source}
\inputencoding{utf8}
\HeaderA{recode}{Recode}{recode}
%
\begin{Description}\relax
Utility function that recodes variables based on user recode specifications.
Handles both numeric or factor variables.
\end{Description}
%
\begin{Usage}
\begin{verbatim}
recode(var, recodes, afr, anr = TRUE, levels)
\end{verbatim}
\end{Usage}
%
\begin{Arguments}
\begin{ldescription}
\item[\code{var}] Variable to recode

\item[\code{recodes}] Character string of recode specifications:
\begin{itemize}

\item Recode specifications in a character string separated by
semicolons of the form \code{input=output} as in:
\code{"1=1;2=1;3:6=2;else=NA"}

\\{}
\item If an input value satisfies more than one specification, then the
first (reading from left to right) is applied

\\{}
\item If no specification is satisfied, then the input value is carried
over to the result unchanged

\\{}
\item \code{NA} is allowed on both input and output

\\{}
\item The following recode specifications are supported:

\\{}

\Tabular{lll}{
\strong{Specification} & \strong{Example}          & \strong{Notes}                                                 \\{}
Single values          & \code{9=NA}               &                                                                \\{}
Set of values          & \code{c(1,2,5)=1}         & The left-hand-side is any R function call that returns a vector\\{}
& \code{seq(1,9,2)='odd'}   &                                                                \\{}
& \code{1:10=1}             &                                                                \\{}
Range of values        & \code{7:9=3}              & Special values \code{lo} and \code{hi} may be used                           \\{}
& \code{lo:115=1}           &                                                                \\{}
Other values           & \code{else=NA}            &
}

\\{}
\item Character values are quoted as in :

\code{recodes = "c(1,2,5)='sanitary' else='unsanitary'"}

\\{}
\item The output may be the (scalar) result of a function call as in:

\code{recodes = "999=median(var, na.rm = TRUE)"}

\\{}
\item Users are advised to carefully check the results of \code{recode()} calls
with any outputs that are the results of a function call.

\\{}
\item The output may be the (scalar) value of a variable as in:

\code{recodes = "999=scalarVariable"}

\\{}
\item If all of the output values are numeric, and if \code{'afr'} is \code{FALSE},
then a numeric result is returned; if \code{var} is a factor then
(by default) so is the result.

\end{itemize}


\item[\code{afr}] Return a factor. Default is TRUE if \code{var} is a factor and is
FALSE otherwise

\item[\code{anr}] Coerce result to numeric (default is TRUE)

\item[\code{levels}] Order of the levels in the returned factor; the default is to use
the sort order of the level names.
\end{ldescription}
\end{Arguments}
%
\begin{Value}
Recoded variable
\end{Value}
%
\begin{Examples}
\begin{ExampleCode}
# Recode values from 1 to 9 to varios specifications
var <- sample(x = 1:9, size = 100, replace = TRUE)

# Recode single values
recode(var = var, recodes = "9=NA")

# Recode set of values
recode(var = var, recodes = "c(1,2,5)=1")

# Recode range of values
recode(var = var, recodes = "1:3=1;4:6=2;7:9=3")

# Recode other values
recode(var = var, recodes = "c(1,2,5)=1;else=NA")

\end{ExampleCode}
\end{Examples}
\inputencoding{utf8}
\HeaderA{villageData}{Cluster Population Weights Dataset}{villageData}
\keyword{datasets}{villageData}
%
\begin{Description}\relax
Dataset containing cluster population weights for use in performing
posterior weighting with the blocked weighted bootstrap approach. This
dataset is from a mother and child health and nutrition survey conducted in
4 districts from 3 regions in Somalia.
\end{Description}
%
\begin{Usage}
\begin{verbatim}
villageData
\end{verbatim}
\end{Usage}
%
\begin{Format}
A data frame with 6 columns and 117 rows:
\begin{description}

\item[\code{region}] Region in Somalia from which the cluster belongs to
\item[\code{district}] District in Somalia from which the cluster belongs to
\item[\code{psu}] The PSU identifier. This must use the same coding system
used to identify the PSUs that is used in the indicators dataset
\item[\code{lon}] Longitude coordinate of the cluster
\item[\code{lat}] Latitude coordinate of the cluster
\item[\code{pop}] Population size of the cluster

\end{description}
\end{Format}
%
\begin{Source}\relax
Mother and child health and nutrition survey in 3 regions of Somalia
\end{Source}
\printindex{}
\end{document}
